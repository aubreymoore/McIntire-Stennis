% Options for packages loaded elsewhere
\PassOptionsToPackage{unicode}{hyperref}
\PassOptionsToPackage{hyphens}{url}
%
\documentclass[
]{article}
\usepackage{amsmath,amssymb}
\usepackage{lmodern}
\usepackage{iftex}
\ifPDFTeX
  \usepackage[T1]{fontenc}
  \usepackage[utf8]{inputenc}
  \usepackage{textcomp} % provide euro and other symbols
\else % if luatex or xetex
  \usepackage{unicode-math}
  \defaultfontfeatures{Scale=MatchLowercase}
  \defaultfontfeatures[\rmfamily]{Ligatures=TeX,Scale=1}
\fi
% Use upquote if available, for straight quotes in verbatim environments
\IfFileExists{upquote.sty}{\usepackage{upquote}}{}
\IfFileExists{microtype.sty}{% use microtype if available
  \usepackage[]{microtype}
  \UseMicrotypeSet[protrusion]{basicmath} % disable protrusion for tt fonts
}{}
\makeatletter
\@ifundefined{KOMAClassName}{% if non-KOMA class
  \IfFileExists{parskip.sty}{%
    \usepackage{parskip}
  }{% else
    \setlength{\parindent}{0pt}
    \setlength{\parskip}{6pt plus 2pt minus 1pt}}
}{% if KOMA class
  \KOMAoptions{parskip=half}}
\makeatother
\usepackage{xcolor}
\IfFileExists{xurl.sty}{\usepackage{xurl}}{} % add URL line breaks if available
\IfFileExists{bookmark.sty}{\usepackage{bookmark}}{\usepackage{hyperref}}
\hypersetup{
  hidelinks,
  pdfcreator={LaTeX via pandoc}}
\urlstyle{same} % disable monospaced font for URLs
\setlength{\emergencystretch}{3em} % prevent overfull lines
\providecommand{\tightlist}{%
  \setlength{\itemsep}{0pt}\setlength{\parskip}{0pt}}
\setcounter{secnumdepth}{-\maxdimen} % remove section numbering
\ifLuaTeX
  \usepackage{selnolig}  % disable illegal ligatures
\fi

\author{}
\date{}

\begin{document}

\hypertarget{mcintire-stennis-progress-report-for-fy2021}{%
\section{McIntire-Stennis Progress Report for
FY2021}\label{mcintire-stennis-progress-report-for-fy2021}}

\hypertarget{guam-forest-biodiversity-inventory}{%
\section{Guam Forest Biodiversity
Inventory}\label{guam-forest-biodiversity-inventory}}

\hypertarget{accomplishments}{%
\subsection{Accomplishments}\label{accomplishments}}

\hypertarget{major-goals-of-the-project}{%
\subsubsection{Major goals of the
project}\label{major-goals-of-the-project}}

\hypertarget{goal-liberate-data-from-biological-collections-and-the-scientific-literature}{%
\paragraph{1 Goal: Liberate data from biological collections and the
scientific
literature}\label{goal-liberate-data-from-biological-collections-and-the-scientific-literature}}

\hypertarget{objective-complete-digitization-of-the-uog-insect-collection}{%
\subparagraph{1.1 Objective: Complete digitization of the UOG insect
collection}\label{objective-complete-digitization-of-the-uog-insect-collection}}

The UOG insect collection catalog has already been made available online
using Symbiota. Note that Symbiota automatically uploads data to GBIF.
The next phase of this digitization project will be imaging of all taxa
in the collection. Existing images will be uploaded and linked to
specimen data. Images will be made for taxa which have not been
previously imaged and these will also be uploaded.

\hypertarget{objective-complete-digitization-of-the-uog-herbarium}{%
\subparagraph{1.2 Objective: Complete digitization of the UOG
herbarium}\label{objective-complete-digitization-of-the-uog-herbarium}}

Digital images are available for all herbarium sheets. The existing
herbarium catalog will be converted from a local database to an online
database using Symbiota or Specify. Both of these online collection
database managers automatically upload to GBIF.

\hypertarget{objective-liberate-data-from-the-scientific-literature}{%
\subparagraph{1.3 Objective: Liberate data from the scientific
literature}\label{objective-liberate-data-from-the-scientific-literature}}

The PI will organize extraction of Guam biodiversity information from
primary scientific publications, starting with Insects of Guam I and II.

\hypertarget{goal-provide-public-access-to-guam-forest-biodiversity-data}{%
\paragraph{2 Goal: Provide public access to Guam forest biodiversity
data}\label{goal-provide-public-access-to-guam-forest-biodiversity-data}}

\hypertarget{objective-build-the-guam-forest-biodiversity-web-site}{%
\subparagraph{2.1 Objective: Build the Guam Forest Biodiversity Web
Site}\label{objective-build-the-guam-forest-biodiversity-web-site}}

The PI will launch a web site to serve as a portal to Guam forest
biodiversity data stored in GBIF. Pages will be developed to dynamically
generate lists such as those suggested above.

\hypertarget{goal-foster-public-interest-in-guams-forest-biodiversity}{%
\paragraph{3 Goal: Foster public interest in Guam's forest
biodiversity}\label{goal-foster-public-interest-in-guams-forest-biodiversity}}

\hypertarget{objective-outreach-and-citizen-science-activities}{%
\subparagraph{3.1 Objective: Outreach and Citizen Science
Activities}\label{objective-outreach-and-citizen-science-activities}}

\begin{itemize}
\tightlist
\item
  The PI will offer annual workshops on the use of iNaturalist, a social
  networking app used by citizen scientists and naturalists which
  enables them to record biodiversity observations with images and
  georeferencing using smart phones. iNaturalist data which is validated
  as research grade by the community is automatically uploaded to GBIF.
\item
  The PI will continue to maintain an iNaturalist project entitled
  Insects of Micronesia.
\item
  The PI will work with the UOG Center for Island Sustainability to
  organize annual bioblitzes. A bioblitz is is an intense period of
  biological surveying in an attempt to record all the living species
  within a designated area. Participants in the bioblitzes will be
  trained to use iNaturalist which will be used to document results.
\end{itemize}

\hypertarget{goal-foster-collaboration-to-help-overcome-the-taxonomic-impediment}{%
\paragraph{4 Goal: Foster collaboration to help overcome the taxonomic
impediment}\label{goal-foster-collaboration-to-help-overcome-the-taxonomic-impediment}}

\hypertarget{objective-collaboration-with-taxonomists-collectors-and-the-biodiversity-informatics-community}{%
\subparagraph{4.1 Objective: Collaboration with taxonomists, collectors
and the biodiversity informatics
community}\label{objective-collaboration-with-taxonomists-collectors-and-the-biodiversity-informatics-community}}

\begin{itemize}
\tightlist
\item
  Collaboration with taxonomists will be cultivated to help identify a
  large backlog of unidentified specimens in the UOG insect collection.
\item
  Existing collaboration will be maintained with existing partners list
  in the Collaboration/Cooperation section of this proposal.
\item
  The PI will participate in at least one scientific meeting per year
  covering biological collections and/or biodiversity informatics.
\item
  The PI will encourage donation of voucher specimens to the UOG insect
  collection from biological surveys such as those being conducted by
  the Ecology of Bird Loss and the baseline surveys being done by
  military contractors in support of the military buildup.
\end{itemize}

\hypertarget{what-was-accomplished-under-these-goals}{%
\subsubsection{What was accomplished under these
goals?}\label{what-was-accomplished-under-these-goals}}

\hypertarget{goal-liberate-data-from-biological-collections-and-scientific-literature}{%
\paragraph{1 Goal: Liberate data from biological collections and
scientific
literature}\label{goal-liberate-data-from-biological-collections-and-scientific-literature}}

We are datamining legacy literature about insects on Guam contained in
Insects of Guam I and II. These volumes document insects collected
during a comprehensive entomological survey of Guam done in 1936.

All 37 chapters of Guam I and II have been datamined by Plazi with
resultant datasets published in Zenodo, Treatment Bank and GBIF (See the
\emph{Other products} section for a chapter list). We are currently
enhancing materials citations annotation by extracting detailed data for
each specimen or series examined in each species treatment section
within each chapter. We have completed 12 chapters and are tracking our
progress using an online status report at
https://aubreymoore.github.io/data-mining-insects-of-guam/MatCit-Validator/status\_report.html.

Insect occurrence records for Guam continue to accumulate in the Global
Biodiversity Information Facility (GBIF). Most records are from data
sources being built and maintained by this project.

GBIF.org (28 December 2020) GBIF Occurrence Download
https://doi.org/10.15468/dl.yfwjwt

\begin{itemize}
\tightlist
\item
  18,604 Guam insect occurrence records
\item
  15,136 records were sourced from the University of Guam Insect
  Collection online catalog.
\item
  457 records were sourced from iNaturalist research-grade observations.
\end{itemize}

GBIF.org (17 December 2021) GBIF Occurrence Download
https://doi.org/10.15468/dl.34ugmb

\begin{itemize}
\tightlist
\item
  19,187 Guam occurrence records
\item
  15,147 records were sourced from the University of Guam Insect
  Collection online catalog.
\item
  776 records were sourced from iNaturalist research-grade observations
\end{itemize}

\hypertarget{goal-provide-public-access-to-guam-forest-biodiversity-data-1}{%
\paragraph{2 Goal: Provide public access to Guam forest biodiversity
data}\label{goal-provide-public-access-to-guam-forest-biodiversity-data-1}}

Please see the section entitled \emph{How have the results been
disseminated to communities of interest}

\hypertarget{goal-foster-public-interest-in-guams-forest-biodiversity-1}{%
\paragraph{3 Goal: Foster public interest in Guam's forest
biodiversity}\label{goal-foster-public-interest-in-guams-forest-biodiversity-1}}

The PI participated in a workshop for educators sponsored by the Guam
Soil and Water Conservation districts entitled \emph{Healthy Forests,
Healthy Communities}.

The PI participated in making a video recording about Guam's forests:

Ares, Adrian. 2021. Video: Forests of Guam. Presentation for the 15th
World Forestry Conference. Western Pacific Tropical Research Center,
University of Guam. Accessed July 27, 2021.
https://www.youtube.com/watch?v=27D-ovSzLBk.

\hypertarget{goal-foster-collaboration-to-help-overcome-the-taxonomic-impediment-1}{%
\paragraph{4 Goal: Foster collaboration to help overcome the taxonomic
impediment}\label{goal-foster-collaboration-to-help-overcome-the-taxonomic-impediment-1}}

Nothing to report.

\hypertarget{what-opportunities-for-training-and-professional-development-has-the-project-provided}{%
\subsubsection{What opportunities for training and professional
development has the project
provided?}\label{what-opportunities-for-training-and-professional-development-has-the-project-provided}}

Plazi provided online training sessions specifically for this project.
Topics covered where \emph{Introduction to Golden Gate Imagine Software}
and \emph{Enhancing Material Citations}. See
https://osf.io/f498p/wiki/home/ for details.

\hypertarget{how-have-the-results-been-disseminated-to-communities-of-interest}{%
\subsubsection{How have the results been disseminated to communities of
interest?}\label{how-have-the-results-been-disseminated-to-communities-of-interest}}

\hypertarget{data-from-the-scientific-literature}{%
\paragraph{Data from the Scientific
literature}\label{data-from-the-scientific-literature}}

For each chapter we annotate, the Plazi workflow creates a journal
article published in Zenodo. This is essentially a republished copy of
the original chapter with links to files containing extracted data.
Links are provided for a GBIF checklist dataset which may be downloaded
in several formats including Darwin core archive (DwCA). In addition, a
dataset for each taxon within the chapter is created and stored in the
Plazi Treatment Bank.

For example, here are the publicly accessible online data products which
were automatically appeared on the internet after we annotated the
\emph{Bees of Guam} chapter:

\textbf{Zenodo article:} Cockerell, T. D. A. (1942). Bees of Guam. In
Insects of Guam I (pp.~188--190). Bernice P. Bishop Museum.
https://doi.org/10.5281/zenodo.5160372

\textbf{GBIF dataset:} Cockerell T D A, carolina (1942). Bees of Guam.
Plazi.org taxonomic treatments database. Checklist dataset
https://doi.org/10.5281/zenodo.5160372 accessed via GBIF.org on
2021-12-17. This dataset can be downloaded in several formats including
Darwin core archive. Note that information within this dataset will be
accessed whenever GBIF is queried.

\textbf{Taxon treatments:} A page is available for each taxon which
appears in the chapter:

\begin{itemize}
\item
  Cockerell, T. D. A. (1942). Apis mellifera Linnaeus. In Bees of Guam,
  pp.~188-190 in Insects of Guam I (p.~188). Bernice P. Bishop Museum.
  https://doi.org/10.5281/zenodo.5211876
\item
  Cockerell, T. D. A. (1942). Megachile laticeps Smith. In Bees of Guam,
  pp.~188-190 in Insects of Guam I (p.~188). Bernice P. Bishop Museum.
  https://doi.org/10.5281/zenodo.5164364
\item
  And so on for the 7 species in this chapter.
\end{itemize}

\hypertarget{data-from-the-university-of-guam-insect-collection}{%
\paragraph{Data from the University of Guam insect
collection}\label{data-from-the-university-of-guam-insect-collection}}

Data and images for each specimen accessioned by the UOG insect
collection are uploaded to a publicly available online database at
https://scan-bugs.org/portal/collections/misc/collprofiles.php?collid=180
which is hosted by the Symbiota Collections of Arthropods Network. All
records are then automatically published on GBIF in a dataset at
https://www.gbif.org/dataset/56e311e3-43c6-4b99-aa21-af396074d5e3.

\hypertarget{what-do-you-plan-to-do-during-the-next-reporting-period-to-accomplish-the-goals}{%
\subsubsection{What do you plan to do during the next reporting period
to accomplish the
goals?}\label{what-do-you-plan-to-do-during-the-next-reporting-period-to-accomplish-the-goals}}

In the last year of this project annotation of materials citations in
Insects of Guam I and II will be completed and a journal article will be
prepared documenting progress that has been made towards realization of
a terrestrial biodiversity inventory for Guam.

\hypertarget{participants}{%
\subsubsection{Participants}\label{participants}}

For Objective 1.3, \emph{Liberate data from the scientific literature},
we are collaborating with Plazi (https://en.wikipedia.org/wiki/Plazi) a
Swiss-based international non-profit association supporting and
promoting the development of persistent and openly accessible digital
bio-taxonomic literature. We work closely with data scientists at the
Plazi office in Brazil: Marcus Guidoti, Carolina Sokolowicz, and Tatiana
Ruschel. They have provided online training, specialized software,
server access, and technical support.

Annette Kang, a PhD student studying entomology at Cornell University,
was hired as a project intern to work on Objective 1.3.

\hypertarget{target-audience}{%
\subsection{Target Audience}\label{target-audience}}

The target audience for data resulting from this project is anybody
interested in Guam's biodiversity. This includes biologists, invasive
species specialists, ecologists, resource planners, biosecurity
officials, students, and the general public.

%\hypertarget{productstype-status-year-published-nifa-support-acknowledged}{%
%\subsection{Products
Type Status Year Published NIFA Support Acknowledged}\label{productstype-status-year-published-nifa-support-acknowledged}}
%
%None.

\hypertarget{citation}{%
\subsection{Citation}\label{citation}}

Moore, Aubrey and Annette Kang 2021. Datamining Insects of Guam. Open
Science Framework. https://osf.io/f498p/

Moore, Aubrey 2021. Datamining Insects of Guam. GitHub repository.
https://aubreymoore.github.io/data-mining-insects-of-guam/

Moore, Aubrey 2021. Insects of Guam Datamining Project Status Report.
https://aubreymoore.github.io/data-mining-insects-of-guam/MatCit-Validator/status\_report.html

\hypertarget{other-products}{%
\subsection{Other Products}\label{other-products}}

Chapters of Insects of Guam I and II which have been datamined with
resultant datasets published in Zenodo, Treatment Bank and GBIF.

\begin{enumerate}
\def\labelenumi{\arabic{enumi}.}
\tightlist
\item
  Cockerell TDA. Halictine Bees from Rota Island. In: Insects of Guam I
  {[}Internet{]}. Honolulu, Hawaii: Bernice P. Bishop Museum; 1942
  {[}cited 2021 Dec 13{]}. p.~191--4. Available from:
  https://zenodo.org/record/5160456
\item
  Moulton D. Thysanoptera: Thrips of Guam. In: Insects of Guam I
  {[}Internet{]}. Bernice P. Bishop Museum Bulletin; 1942 {[}cited 2021
  Dec 13{]}. p.~7--16. Available from: https://zenodo.org/record/3634035
\item
  Swezey OH. Lepidoptera, Geometridae, Arctiidae, Agrotidae,and
  Pyralidae of Guam. In: Insects of Guam II {[}Internet{]}. Honolulu,
  Hawaii: Bernice P. Bishop Museum Bulletin 189; 1942 {[}cited 2021 Dec
  13{]}. p.~163--85. Available from: https://zenodo.org/record/5165313
\item
  Swezey OH. Orthoptera And Related Orders Orthoptera And Related Orders
  Of Guam. In: Insects of Guam II {[}Internet{]}. Honolulu, Hawaii:
  Bernice P. Bishop Museum; 1946 {[}cited 2021 Dec 13{]}. p.~3--8.
  Available from: https://zenodo.org/record/5160233
\item
  SWEZEY OH, WILLIAMS FX. ODONATA, DRAGONFLIES OF GUAM. In: Insects of
  Guam I {[}Internet{]}. Honolulu, Hawaii: Bernice P. Bishop Museum,
  Bulletin 172; 1942 {[}cited 2021 Dec 13{]}. p.~3--6. Available from:
  https://zenodo.org/record/5159515
\item
  Mailoch JR. Trypetidae, Otitidae, Helomyzidae, And Clusiidae of Guam
  (Diptera). In: Insects of Guam I {[}Internet{]}. Honolulu, Hawaii:
  Bernice P. Bishop Museum; 1942 {[}cited 2021 Dec 13{]}. p.~201--10.
  Available from: https://zenodo.org/record/5163626
\item
  Johannse OA. Some New Species Of Nemocerous Diptera From Guam. In:
  Insects of Guam II {[}Internet{]}. Honolulu, Hawaii: Bernice P. Bishop
  Museum; 1946 {[}cited 2021 Dec 13{]}. p.~187--93. Available from:
  https://zenodo.org/record/5169292
\item
  Gressitt JL. New Longicorn Beetles From Guam (Cerambycidae). In:
  Insects of Guam I {[}Internet{]}. Honolulu, Hawaii: Bernice P. Bishop
  Museum; 1942 {[}cited 2021 Dec 13{]}. p.~61--4. Available from:
  https://zenodo.org/record/5159791
\item
  Van Zwaluwenburg RH. Elaterid And Eucnemid Beetles Of Guam. In:
  Insects of Guam I {[}Internet{]}. Honolulu, Hawaii: Bernice P. Bishop
  Museum; 1942 {[}cited 2021 Dec 13{]}. p.~53--5. Available from:
  https://zenodo.org/record/5159555
\item
  Metcalf ZP. Homoptera, Fulgoroidea and Jassoidea of Guam. In: Insects
  of Guam II {[}Internet{]}. Honolulu, Hawaii: Bernice P. Bishop Museum
  Bulletin 189; 1946 {[}cited 2021 Dec 13{]}. p.~105--48. Available
  from: https://zenodo.org/record/5174008
\item
  Swezey OH. Notes On Some Fulgoroidea Of Guam. In: Insects of Guam II
  {[}Internet{]}. Honolulu, Hawaii: Bernice P. Bishop Museum; 1946
  {[}cited 2021 Dec 13{]}. p.~149--56. Available from:
  https://zenodo.org/record/5164064
\item
  Fullaway DT. Ichneumonidae, Evaniidae, And Braconidae Of Guam. In:
  Insects of Guam II {[}Internet{]}. Honolulu, Hawaii: Bernice P. Bishop
  Museum; 1946 {[}cited 2021 Dec 13{]}. p.~221--7. Available from:
  https://zenodo.org/record/5156759
\item
  Usinger RL. Hemiptera Heteroptera of Guam. In: Insects of Guam II
  {[}Internet{]}. Honolulu, Hawaii: Bernice P. Bishop Museum, Bulletin
  189; 1946 {[}cited 2021 Dec 13{]}. p.~11--103. Available from:
  https://zenodo.org/record/5173934
\item
  Fullaway DT. Hymenoptera, New Species Of Guam Chalcidoidea. In:
  Insects of Guam II {[}Internet{]}. Honolulu, Hawaii: Bernice P. Bishop
  Museum; 1946 {[}cited 2021 Dec 13{]}. p.~201--10. Available from:
  https://zenodo.org/record/5169330
\item
  Blair KG. Coleoptera Heteromera From Guam. In: Insects of Guam I
  {[}Internet{]}. Honolulu, Hawaii: Bernice P. Bishop Museum; 1942
  {[}cited 2021 Dec 13{]}. p.~56--60. Available from:
  https://zenodo.org/record/5159673
\item
  Swezey OH. Miscellaneous Families of Guam Coleoptera. In: Insects of
  Guam I {[}Internet{]}. Honolulu, Hawaii: Bernice P. Bishop Museum;
  1942 {[}cited 2021 Dec 13{]}. p.~150--71. Available from:
  https://zenodo.org/record/5167701
\item
  Swezey OH. Aphididae and Aleurodidae Of Guam. In: Insects of Guam I
  {[}Internet{]}. Honolulu, Hawaii: Bernice P. Bishop Museum; 1942
  {[}cited 2021 Dec 13{]}. p.~23--4. Available from:
  https://zenodo.org/record/5159809
\item
  Swezey OH. SOME MISCELLANEOUS DIPTERA OF GUAM. In: Insects of Guam-II
  {[}Internet{]}. Hawaii, Honolulu: Bernice P. Bishop Museum; 1946
  {[}cited 2021 Dec 13{]}. p.~195--200. Available from:
  https://zenodo.org/record/5127686
\item
  LALLEMAND V. Homoptera, Cercopidae of Guam. In: Insects of Guam I
  {[}Internet{]}. Honolulu, Hawaii: Bernice P. Bishop Museum, Bulletin
  172; 1942 {[}cited 2021 Dec 13{]}. p.~17--8. Available from:
  https://zenodo.org/record/5159714
\item
  Alexander CP. Diptera, Tipulidae of Guam. In: Insects of Guam I
  {[}Internet{]}. Honolulu. Hawaii: Bernice P. Bishop Museum, Bulletin,
  172; 1942 {[}cited 2021 Dec 13{]}. p.~195--8. Available from:
  https://zenodo.org/record/5174000
\item
  Fullaway DT. Coccidae of Guam. In: Insects of Guam II {[}Internet{]}.
  Honolulu, Hawaii: Bernice P. Bishop Museum Bulletin 189; 1946 {[}cited
  2021 Dec 13{]}. p.~157--62. Available from:
  https://zenodo.org/record/5164252
\item
  CockerellL TDA. Bees of Guam. In: Insects of Guam I {[}Internet{]}.
  Honolulu, Hawaii: Bernice P. Bishop Museum; 1942 {[}cited 2021 Dec
  13{]}. p.~188--90. Available from: https://zenodo.org/record/5160372
\item
  Swezey OH. Hymenoptera Formicidae of Guam. In: Insects of Guam I
  {[}Internet{]}. Honolulu, Hawaii: Bernice P. Bishop Museum; 1942
  {[}cited 2021 Dec 13{]}. p.~175--83. Available from:
  https://zenodo.org/record/5160270
\item
  Swezry OH. Sphingidae Of Guam. In: Insects of Guam I {[}Internet{]}.
  Honolulu, Hawaii: Bernice P. Bishop Museum; 1942 {[}cited 2021 Dec
  13{]}. p.~39--40. Available from: https://zenodo.org/record/5160080
\item
  Zimmerman EC. Rhipiceridae Of Guam. In: Insects of Guam I
  {[}Internet{]}. Honolulu, Hawaii: Bernice P. Bishop Museum; 1942
  {[}cited 2021 Dec 13{]}. p.~45--6. Available from:
  https://zenodo.org/record/5159434
\item
  Swezey OH. Strepsiptera of Guam. In: Insects of Guam I {[}Internet{]}.
  Honolulu, Hawaii: Bernice P. Bishop Museum; 1942 {[}cited 2021 Dec
  13{]}. p.~173--173. Available from: https://zenodo.org/record/5160090
\item
  SchedL KE. Barkbeetles of Guam. In: Insects of Guam I {[}Internet{]}.
  Honolulu, Hawaii: Bernice P. Bishop Museum; 1942 {[}cited 2021 Dec
  13{]}. p.~147--9. Available from: https://zenodo.org/record/5160072
\item
  Zimmerman EC. Curculionidae of Guam. In: Insects of Guam I
  {[}Internet{]}. Honolulu, Hawaii: Bernice P. Bishop Museum; 1942
  {[}cited 2021 Dec 13{]}. p.~73--146. Available from:
  https://zenodo.org/record/5159964
\item
  Swezey OH. Wasps of Guam. In: Insects of Guam I {[}Internet{]}.
  Honolulu, Hawaii: Bernice P. Bishop Museum; 1942 {[}cited 2021 Dec
  13{]}. p.~184--7. Available from: https://zenodo.org/record/5160297
\item
  Swezey OH. Lepidoptera, Butterflies of Guam. In: Insects of Guam I
  {[}Internet{]}. Honolulu, Hawaii: Bernice P. Biship Museum; 1942
  {[}cited 2021 Dec 13{]}. p.~31--8. Available from:
  https://zenodo.org/record/5160043
\item
  Zimmerman EC. Anthribidae Of Guam. In: Insects of Guam I
  {[}Internet{]}. Honolulu, Hawaii: Bernice P. Bishop Museum; 1942
  {[}cited 2021 Dec 13{]}. p.~65--72. Available from:
  https://zenodo.org/record/5159835
\item
  Swezey OH. Culicidae of Guam. In: Insects of Guam I {[}Internet{]}.
  Honolulu, Hawaii: Bernice P. Bishop Museum; 1942 {[}cited 2021 Dec
  13{]}. p.~199--200. Available from: https://zenodo.org/record/5173998
\item
  Zimmerman EC. Ciidae of Guam. In: Insects of Guam-I {[}Internet{]}.
  Honolulu, Hawaii: Bernice P. Bishop Museum; 1942 {[}cited 2021 Dec
  13{]}. p.~47--52. Available from: https://zenodo.org/record/5159455
\item
  Swezey OH. Membracidae of Guam. In: Insects of Guam I {[}Internet{]}.
  Honolulu, Hawaii: Bernice P, Bishop Museum; 1942 {[}cited 2021 Dec
  13{]}. p.~19. Available from: https://zenodo.org/record/5159743
\item
  Bernhauer M. Coleoptera, Staphylinidae Of Guam. In: Insects of Guam I
  {[}Internet{]}. Bernice P. Bishop Museum; 1942 {[}cited 2021 Dec
  13{]}. p.~41--4. Available from: https://zenodo.org/record/5159420
\item
  Light SF. Isoptera of Guam. In: Insects of Guam II {[}Internet{]}.
  Bernice P. Bishop Museum; 1946 {[}cited 2021 Dec 13{]}. p.~9--9.
  Available from: https://zenodo.org/record/5160243
\item
  Banks N. Neuropteroid Insects from Guam. In: Insects of Guam I
  {[}Internet{]}. Honolulu, Hawaii: Bernice P. Bishop Museum; 1942
  {[}cited 2021 Dec 13{]}. p.~25--30. Available from:
  https://zenodo.org/record/5159923
\end{enumerate}

\hypertarget{changesproblems}{%
\subsection{Changes/Problems}\label{changesproblems}}

The University of Guam insect collection has been mothballed in a small
storage room with poor environmental conditions and insufficient work
space.

\end{document}
